\documentclass[mathserif]{beamer}
\usepackage[latin1]{inputenc}
\usepackage{graphicx}
\usepackage{hyperref}
%\usepackage{qtree}
%\usepackage{amssymb, amsthm}
\usetheme{default}
\title[sX Maker Crash Course]{Development Crash Course\\ \large Development Environment and git}
\author{Charles Julian Knight}
\institute{Startup Exchange - Georgia Tech}
\date{\today}
\begin{document}

\begin{frame}
\titlepage
\end{frame}

\begin{frame}{Overview}
\begin{itemize}
\item sX Makers
\item Crash course
\item Tips, Tools, and Traps
\item Variables, conditionals, loops, arrays?
\pause \item You already know how to code*
\item Syntax and algorithms
\end{itemize}
\end{frame}

\begin{frame}{Where do I start?}
Development Environment\\
\vspace{.5cm}
\begin{itemize}
\item Unix-like (Linux or OSX)
\item Why not Windows?
\item Virtualbox
\item Ubuntu
\item Command line introduction:\\ {\tiny \url{http://vic.gedris.org/Manual-ShellIntro/1.2/ShellIntro.pdf}}
\end{itemize}
\end{frame}

\begin{frame}{git and GitHub}
git\\
\begin{itemize}
\item SCM and version control
\item History: 2005
\item kernel
\item \texttt{sudo apt-get install git}
\end{itemize}
\vspace{.3in}
\pause
GitHub\\
\begin{itemize}
\item Cloud repositories
\item Startup, 2011
\item restrictions
\end{itemize}
\end{frame}

\begin{frame}{Vocabulary}
\begin{itemize}
\item repository - a place to store code
\item commit - changes to the code
\item branch - separate series of commits
\item clone vs. fetch vs. pull
\begin{itemize}
  \item clone - download and create a local repository
  \item fetch - download current branches to existing local repository
  \item pull - download the current remote branch and merge it with current local branch
\end{itemize}
\item fork - create a new repository based on existing
\item {\url{http://git-scm.com/book/en/}}
\end{itemize}
\end{frame}

\begin{frame}{Basic Usage}
\texttt{\small
git clone https://github.com/<username>/<repository>.git\\
git checkout -b <branch\_name>\\
<make changes>\\
git status\\
git add .\\
git commit\\
git merge <branch\_name>\\
git push origin master\\
}
\vspace{.3in}
\url{http://try.github.io}
\end{frame}

\end{document}
